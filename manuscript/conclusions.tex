\section{Conclusions \& Discussion}
\label{sec:conclusions}

In this Letter, we examined Moosinesq convection.
We showed that convection inside moose produces interesting flow morphologies, and we noted that these flow complexities may have important implications for the evolution of moosive stars.
Whether Moosinesq convection drives episodic shedding which can impact the resulting moosive (stellar) tracks remains an exciting avenue of future exploration.
While moose are solitary animals, moosive stars tend to come in pairs \citep{goodwin_kroupa_2005}, and it is unclear how binary moose interactions\footnote{Fights.}  would affect dynamics or observables.

We examined a simple simulation with a Prantler number of unity; future work should lower the Prantler number, which would be more appropriate for stellar interiors \citep{garaud_2021} and could probably be achieved by studying a younger moose.
Moosive stars rotate rapidly, so future authors should study rotating convection at low Elkman number.
We suggest quantitative comparison to Rayleigh-B\'{e}nard convection through computation of the Mooselt number Moo, to understand how it compares to the classical Nusselt number Nu.
Our results are also limited by the single drawing of a moose we were able to obtain.
Future work in this young field of Moosinesq convection should consider the impact of moose depicted in different poses or drawn in different styles.
This is a promising area of study for early-career scientists\footnote{(9 and under)} \cite[see e.g.,][Fig.~9]{luger_etal_2019}.

In this work we have focused on astrophysical applications, but the methods developed in this work may provide a pathway toward unraveling the mysteries of other wildlife-related fluid phenomena, such as the powerful and mysterious otter of \cite{Schwab2021}\footnote{Also known as the Papaloizou-Pringle Patronus.}.
Lessons learned from this and future work on the Moosinesq approximation may also be of interest to those working in the yet-underexplored field of \textit{Goosinesq}\footnote{We leave the definition of this approximation to the reader's first thought.} convection.

The Boussinesq and Moosinesq approximations are formally valid when all length scales in the problem are small compared to the scale height and when compressibility is unimportant.
However, there are various applications in microbiology \citep[e.g.][]{Ravetto2014} and wildlife ecology \citep[e.g.][]{Enright1963} wherein compressibility may be interesting.
Studies have often focused on domains in which animal compressibility is more pronounced, such as deep-sea life.
However, motivated by a recent report of an unexpected occurrence of acute human compression and deformation via moose interaction \citep{Gudmannsson2018}, we recommend that the land-mammal regime of Moosinesq approximation be extended beyond the incompressible constraint studied here.


