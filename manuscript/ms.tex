% Preamble
\documentclass[onecolumn, twocolappendix]{aastex631}
%\documentclass[twocolumn]{aastex631}
\usepackage{natbib}
\usepackage{latexsym}
\usepackage{graphicx}
\usepackage{epsfig}
\usepackage{amssymb}
\usepackage{amsmath}
\usepackage{epstopdf}
\usepackage{hyperref}
\usepackage{xcolor}

%%%% Custom commands
\newcommand{\yL}{\ensuremath{\mathcal{Y}_{\rm{L}}}}
\newcommand{\yS}{\ensuremath{\mathcal{Y}_{\rm{S}}}}
\newcommand{\justgrad}{\ensuremath{\nabla}}
\newcommand{\gradrad}{\ensuremath{\nabla_{\rm{rad}}}}
\newcommand{\gradad}{\ensuremath{\nabla_{\rm{ad}}}}
\newcommand{\gradC}{\ensuremath{\nabla_{\mathrm{C}}}}
\newcommand{\gradmu}{\ensuremath{\nabla_{\mu}}}
\newcommand{\gradL}{\ensuremath{\nabla_{\mathrm{L}}}}
\newcommand{\gradT}{\ensuremath{\nabla_{\mathrm{T}}}}
\newcommand{\Ro}{\ensuremath{\mathrm{R}_{0}}}
\newcommand{\delp}{\ensuremath{\delta_{\rm{p}}}}
\newcommand{\Fbot}{\ensuremath{F_{\rm{bot}}}}
\newcommand{\Ftot}{\ensuremath{F_{\rm{tot}}}}
\newcommand{\Frad}{\ensuremath{F_{\rm{rad}}}}
\newcommand{\Fconv}{\ensuremath{F_{\rm{conv}}}}
\newcommand{\Fcz}{\ensuremath{F_{\rm{cz}}}}
\newcommand{\mP}{\ensuremath{\mathcal{P}}}
\newcommand{\mD}{\ensuremath{\mathcal{D}}}
\newcommand{\dP}{\ensuremath{\delta_{\rm{p}}}}
\newcommand{\Lcz}{\ensuremath{L_{\rm{CZ}}}}
\newcommand{\mR}{\ensuremath{\mathcal{R}}}
\newcommand{\mS}{\ensuremath{\mathcal{S}}}
\newcommand\Pran{\ensuremath{\mathrm{Pr}}}
\newcommand{\brunt}{{Brunt-V\"{a}is\"{a}l\"{a}}}

\newcommand{\angles}[1]{\langle #1 \rangle}
\newcommand{\pd}[1]{\partial_{#1}}
\renewcommand{\vec}[1]{\boldsymbol{#1}}
\newcommand{\M}[1]{\mathbf{#1}}
\renewcommand{\dot}{\vec{\cdot}}
\renewcommand{\bar}[1]{\overline{#1}}
\newcommand{\grad}{\vec{\nabla}}
\newcommand{\cross}{\vec{\times}}
\newcommand{\laplacian}{\nabla^2}

\newcommand{\editone}[1]{\textcolor{orange}{#1}}

%%%% Journal preamble
\received{April 1, 2022}
\revised{April 1, 2022}
\accepted{???}
%\published{April 1, 2022}
\submitjournal{The Journal of Universal Rejection}

\shorttitle{Moosinesq Convection}
\shortauthors{Anders et al}

\begin{document}

%%%% Title and Abstract
\title{Moosinesq Convection in the Cores of Moosive Stars}
\author[0000-0002-3433-4733]{Evan H. Anders}
\affiliation{CIERA, Northwestern University, Evanston IL 60201, USA}
\author[0000-0002-4791-6724]{Evan B. Bauer}
\affiliation{Center for Astrophysics $\vert$ Harvard \& Smithsonian, 60 Garden St., Cambridge, MA 02138, USA}
\author[0000-0001-5048-9973]{Adam S. Jermyn}
\affiliation{Center for Computational Astrophysics, Flatiron Institute, New York, NY 10010, USA}
\author[0000-0001-8935-219X]{Benjamin P. Brown}
\affiliation{Department Astrophysical and Planetary Sciences \& LASP, University of Colorado, Boulder, CO 80309, USA}
\author[0000-0003-1651-9141]{Eric W. Hester}
\affiliation{Department of Mathematics, The University of California, Los Angeles, 520 Portola Plaza 90024, California, United States of America}
\author{Mindy Wilkinson}
\affiliation{Primum Terrae LLC}
\author[0000-0003-1012-3031]{Jared A. Goldberg}
\affiliation{Department of Physics, University of California, Santa Barbara, CA 93106, USA}
\author[0000-0002-4472-8517]{Samuel J. Van Kooten}
\affiliation{Southwest Research Institute, Boulder CO 80302, USA}
\author[0000-0002-7635-9728]{Daniel Lecoanet}
\affiliation{CIERA, Northwestern University, Evanston IL 60201, USA}
\affiliation{Department of Engineering Sciences and Applied Mathematics, Northwestern University, Evanston IL 60208, USA}

\correspondingauthor{Evan H. Anders}
\email{evan.anders@northwestern.edu}

\begin{abstract}
    Stars with masses $\gtrsim$ 1.1$M_\odot$ have core convection zones during their time on the main sequence.
    In these \emph{moosive stars}, convection introduces many uncertainties in stellar modeling.
    In this Letter, we build upon the Boussinesq approximation to present the first-ever simulations of \emph{Moosinesq convection}, which captures the complex geometric structure of the convection zones of these stars.
    These flows are bounded in a manner informed by the majestic terrestrial \emph{Alces alces} (moose) and could have important consequences for the evolution of these stars.
    We find that Moosinesq convection results in very interesting flow morphologies, and that the pun opportunities it provides are seemingly bottomless.
\end{abstract}
\keywords{TODO: something clever?}

%%%% Body of paper
\section{Introduction}
\label{sec:introduction}

Convection is important in all sorts of natural systems\footnote{Trust us, we're experts}.
In particular, most stars have convection zones.
This can be determined by looking at the Sun\footnote{Please do not look at the Sun.} or by running 1D stellar evolution models.
It is crucial that we gain a better understanding of convection since it is such a ubiquitous and poorly understood process in astrophysics\footnote{The financial wellfare of many of the authors depend on you agreeing with this statement.}.

Convective motions are three-dimensional, so many modern convective experiments utilize multi-dimensional fluid dynamical simulations.
However, modern computational resources are too limited to time-evolve the Navier-Stokes equations in their most complex form \citep{landau}, including all of the important physics present in stars \citep{Paxton2011, Paxton2013, Paxton2015, Paxton2018, Paxton2019}.
As a result, numericists make simplifying assumptions to create tractable experiments, for example by simplifying the geometry or the equations.
The most widely-used simplified set of equations is called the \emph{Boussinesq approximation} \citep{spiegel_veronis_1960}, and it has been used in thousands of studies \citep[see e.g.,][]{ahlers_etal_2009}.

In this Letter, we present the first ever simulations which use an oft-overlooked simplified equation set: the \emph{Moosinesq} approximation.
This approximation is similar to the Boussinesq approximation in every way, except it allows for the study of convection inside of a moose (\emph{Alces alces}).
While this approximation has many\footnote{\emph{many}} applications in the terrestrial context, the authors of this paper are largely astronomers.
Therefore we are most interested in the application of this approximation to \emph{moosive stars}, that is, stars with masses $M \gtrsim 1.3 M_\odot$ which have convective course on the main sequence.
In section \ref{sec:methods}, we describe the Moosinesq equations and our numerical methods.
In section \ref{sec:results}, we demonstrate the nature of flows in two-dimensional Moosinesq Convection.
Finally, in section \ref{sec:conclusions}, we discuss applications of Moosinesq Convection to stars, human society, and beyond.


\section{Numerical methods}
\label{sec:methods}

We time-evolve the Moosinesq Equations,
\begin{align}
    \grad\dot\vec{u} &= 0,
    \label{eqn:incompressible}, \\
    \partial_t \vec{u} + \vec{u}\dot\grad\vec{u} &= -\grad \varpi - T\vec{g} + \nu \grad^2 \vec{u} - \gamma \mathcal{M} \vec{u},
    \label{eqn:moosementum}, \\
    \partial_t T + \vec{u}\dot\grad T &= \kappa_T \grad^2 \vec{u} - \gamma \mathcal{M} T,
    \label{eqn:temperature}
\end{align}
which are just the Boussinesq Equations \citep{spiegel_veronis_1960} with crucial Moose $\mathcal{M}$ terms in the moosementum and temperature equations Eqns.~(Eqn.~\ref{eqn:moosementum}- \ref{eqn:temperature}).
Here, $\vec{u}$ is the velocity, $T$ is the temperature, $\varpi$ is the reduced pressure, $\nu$ is the kinematic viscosity, $\kappa_T$ is the thermal diffusivity, and $\gamma$ is a frequency associated with the damping of motions.
We solve these equations in polar $(r, \phi)$ geometry, because we are astrophysicists and this geometry is most applicable to moosive stars.
We naturally choose to have gravity point down in a Cartesian sense, $\vec{g} = - g \hat{z} = - g (\sin\phi \hat{r} + \cos\phi \hat{\phi})$ for increased confusion and lack of clarity.

The Moose is implemented using the volume penalization method described in e.g., \citet{hester_etal_2021}.
We first take an image of a majestic moose from the internet (Fig.~\ref{fig:methods}, left\footnote{Available online at \url{https://www.publicdomainpictures.net/en/view-image.php?image=317077&picture=moose}.}).
We next compute a signed distance function $d_s$ at each pixel to determine how far that pixel is from the edge of the moose.
We convert that signed distance function (whose range is [-0.5, 0.5]) into a profile that varies smoothly from 0 to 1 over the moose boundaries, $\mathcal{M} = 0.5(1 - \mathrm{erf}(\pi^{1/2}d/\delta))$.
We then interpolate from pixel values into polar coordinates ($r$, $\phi$) sampled on the natural grid of our spectral bases (Fig.~\ref{fig:methods}, middle).
The resulting moose mask which is fed directly into our equations during timestepping is thus produced and shown in Fig.~\ref{fig:methods}, right.

\begin{figure*}[t!]
\centering
\includegraphics[width=\textwidth]{paper_figure01.pdf}
    \caption{ 
        (Left) A public-domain silhouette of a moose.
        (Middle) A sparse representation of the polar-coordinate grid on which we represent fields in our simulation.
        (Right) The Moosinesq mask $\mathcal{M}$ felt by our equations; fluid motions are damped where $\mathcal{M} > 0$.
        \label{fig:methods}
    }
\end{figure*}


\section{Results}
\label{sec:results}

We display the majesty of moosinesq convection in Fig.~\ref{fig:paper_figure02.pdf}.
We visualize the temperature field $T$, so red is warm fluid that buoyantly rises, and blue is cold fluid that buoyantly falls.
Plotted over the temperature field is the mask $\mathcal{M}$, which is fully transparent when it is zero and which is a low-opacity white when $\mathcal{M} = 1$.
This allows us to show that, indeed, there are no appreciable motions outside of the moose and the mask is working properly.

The moose is filled with interesting dynamics\footnote{Likely due to a recent, delicious meal.}.
The legs largely serve as thin, tall ``tunnels'' which are filled with Von K\'{a}rm\'{a}n vortices and which connect the hot moose feet to its neutrally-buoyant body.
Cold fluid parcels from the body have managed to mix down one leg, and the moose would probably benefit from having that leg wrapped in a warm cloth or heat pack.
The body of the moose exhibits dynamics familiar from classical 2D Rayleigh-B\`{e}nard convection.
Hot and cold fluid swirl together, forming many vortices, and mix.
Aside from the legs, the antlers are probably the most interesting part of the moose.
Long-lived vortices of relatively hot fluid establish themselves there and turn for a few convective overturn times.
Then, violent flows from the moose's body disrupt those vortices with fresh, hot fluid and this process repeats itself.
Occasionally some of this hot fluid rises into the tips of the moose's antlers, which probably accounts for the growth of the moose's antlers.

\begin{figure*}[tp!]
\centering
    \includegraphics[width=\textwidth]{paper_figure02.pdf}
\caption{ The beautiful, powerful moose.
\label{fig:dynamics}
}
\end{figure*}

Now that we have examined the dynamics in Moosinesq Convection in some detail, we turn our attention towards its astrophysical applications.
\citet{kaiser_etal_2020} note that ``massive [sic] stars'' are sensitive to ``the details of their complex convective history'' \citep{kaiser_etal_2020}.
We agree.
One consideration that previous authors have ignored when considering convective uncertainties in moosive stars is displayed in Fig.~\ref{fig:moosive_stars}.
That is, the cores of these stars are filled with Moosinesq Convection, which can have important consequences for moosive stellar evolution.
it is unclear at this time how the complex flow morphologies associated with moosinesq convection would affect e.g., the magnetic fields, chemical profiles, and lifetimes of these stars.
We leave these important considerations to future work.

\begin{figure*}[t!]
\centering
    \includegraphics[width=\textwidth]{moosive_stars.pdf}
\caption{ Moosive Stars.
\label{fig:moosive_stars}
}
\end{figure*}




\section{Conclusions \& Discussion}
\label{sec:conclusions}




\begin{acknowledgments}
    We thank the Dedalus development team for creating an excellent tool that lets us create \sout{stupid} groundbreaking simulations like the one studied here.
EHA is funded as a CIERA Postdoctoral fellow and would like to thank CIERA and Northwestern University. 
This research was supported in part by the National Science Foundation under Grant No. PHY-1748958, and we acknowledge the hospitality of KITP during the Probes of Transport in Stars Program.
Computations were conducted with support from the NASA High End Computing (HEC) Program through the NASA Advanced Supercomputing (NAS) Division at Ames Research Center on Pleiades with allocation GID s2276.
The Flatiron Institute is supported by the Simons Foundation.
\end{acknowledgments}

\appendix

\section{Historical human-moose interactions}
\label{app:history}

The circumpolar distribution of \textit{Alces alces} (moose) has led to multiple independent points of cultural connection to and subsequent efforts directed at harnessing the largest cervid with varying levels of success.
Rock carvings in the Kalbak-Tash group, Altai Republic, Russia indicate that \textit{ab antuquo} efforts to ride or cause moose to pull sleds have been documented and subsequently received comment since the Bronze Age and likely earlier \citep{useev_2014}.
In North America, European efforts to colonize Canada have been intermittently aided by the capture of wild individual moose more suited to traversing long distances in snow and over boggy ground than domestic alternatives, leading to practical applications such as the use of a team of moose to deliver mail by Mr.\ W.R.\ Day \citep{archives_unleashed}.
The exploitation of this practicality was limited, obviously, by the size and temperament of the moose, which after about the age of 3 days is entirely antagonistic towards humans \citep{sipko_etal_2019}.
A few spectacular exceptions of moose under harness have kept hope burning for a future with a less fraught relationship.
The most noteworthy partnership is widely remembered through the efforts of Albert Vaillancourt of Chelmsford, Ontario, who exhibited moose pulling a surrey during the intermissions of horse racing \citep{chisholm_2019}.
His pair of racing mooses named Moose and Silver clearly demonstrated the majesty and potential of this species \citep{landry_1941}.
 
This potential inspires the continuing quest for a fully domestic moose that is somewhat less likely to attack and kill humans. 
In Russia, moose domestication is the subject of investigations initiated by Prof. P. A. Manteifel, who oversaw an effort to create moose nurseries across Russia for the purpose of creating a recognizably domestic animal from about 1934 through the present \citep{sipko_etal_2019}.
In an early, perhaps premature demonstration on December 1937, I.V. Stalin watched a military moose drill. 
He was “particularly impressed by the moment when moose cavalry flew out of the forest, bristling with machine guns.” He did note that the moose were not yet trained to distinguish the Red Army soldiers from the White Finns \citep{pererva_2017}\footnote{Some internet sources believe that this was not truly said, but today is April 1, so it stands in our manuscript.}.


\section{Moose Mask Creation}
\label{app:mask}

The Moose mask $\mathcal{M}(r,\phi)$ used in Eqns.~\ref{eqn:moosementum} \& \ref{eqn:temperature} and shown in the right panel of Fig.~\ref{fig:methods} is constructed as follows.
We read in the moose image in the left panel of Fig.~\ref{fig:methods}, and read the color value of each pixel.
We then compute a signed distance function $d(x,y)$ with $d(x,y) \in [-0.5, 0.5]$ to determine how far each pixel is from a boundary of the moose, with zero values being at the boundaries.
We next calculate the mask value of each pixel as
\begin{equation}
\mathcal{M}(x,y) = \frac{1}{2}\left(1-\mathrm{erf}\left[\frac{\sqrt{\pi} d(x,y)}{\delta}\right]\right),
\end{equation}
where we choose $\delta = 31.113 \left(\rm{Pr}/\rm{Ra}\right)^{1/4} \tilde{\gamma}^{-1/2}$ (see appendix \ref{app:nondim_equations}), which is ten times larger than the optimal, marginally-resolved $\delta$.
We then interpolate the mask and sample it on our simulation grid in polar ($r, \phi$) coordinates.
The resulting moose mask is used directly during timestepping.
For more specifics, we refer the reader to \url{https://github.com/evanhanders/moosinesq_convection/blob/main/masks/smooth_moosinesq_ibm.ipynb}.


\section{Nondimensional Equations, Simulation Details \& Data Availability}
\label{app:nondim_equations}
We time-evolve a nondimensionalized form of the Moosinesq equations.
We choose the radius of our polar geometry domain as our nondimensional lengthscale $L$.
We choose the temperature difference between the points ($r$, $\phi$) = ($L$, $\pi/2$) \& ($L$, $3\pi/2$) to be the nondimensional temperature scale $\Delta T$.
The freefall velocity is therefore $u_{\rm ff} = \sqrt{\alpha g L \Delta T}$ and the nondimensional timescale is $\tau = L/u_{\rm ff}$.
We furthermore define the Rayleigh and Prantler numbers,
\begin{equation}
\mathrm{Ra} = \frac{\alpha g L^3 \Delta T}{\nu \kappa_T},
\qquad
\mathrm{Pr} = \frac{\nu}{\kappa_T},
\end{equation}
and the nondimensional timescale $\tilde{\gamma} = \gamma\tau$.

The nondimensional Moosinesq equations are then
\begin{align}
    \grad\dot\vec{u} &= 0,
    \label{eqn:nd_incompressible}, \\
    \partial_t \vec{u} + \vec{u}\dot\grad\vec{u} &= -\grad \varpi + T\hat{z} + \sqrt{\frac{\mathrm{Pr}}{\mathrm{Ra}}} \grad^2 \vec{u} - \tilde{\gamma} \mathcal{M} \vec{u},
    \label{eqn:nd_moosementum}, \\
    \partial_t T + \vec{u}\dot\grad T &= \frac{1}{\sqrt{\mathrm{RaPr}}} \grad^2 \vec{u} - \tilde{\gamma} \mathcal{M} T.
    \label{eqn:nd_temperature}
\end{align}
The initial temperature field is linear and is hot at the bottom and cold at the top so that $T(r,\phi) = T_0 = -z/2$, where $z = r\sin(\phi)$
We perturb this initial temperature field with noise whose magnitude is $10^{-3}$ and multiply that noise by $1 - \mathcal{M}$ to start the convective instability.

We time-evolve equations \ref{eqn:nd_incompressible}-\ref{eqn:nd_temperature} using the Dedalus pseudospectral solver \citep[][version 3 on commit c153f2e]{burns_etal_2020} using timestepper RK443 and CFL safety factor 0.4.
The equations are solved on a \texttt{DiskBasis} with 2048 radial and 4096 azimuthal points; this means that all variables are represented using TODO with 2048 radial coefficients and 4096 azimuthal.
To avoid aliasing errors, we use the 3/2-dealiasing rule in all directions.

The Python scripts and Jupyter notebooks used to perform the simulation and create the figures in this paper, are available online at \url{https://github.com/evanhanders/moosinesq_convection}.


\newpage
\bibliographystyle{aasjournal}
\bibliography{biblio}
\end{document}
