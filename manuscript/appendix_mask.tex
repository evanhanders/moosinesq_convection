\section{Moose Mask Creation}
\label{app:mask}

The Moose mask $\mathcal{M}(r,\phi)$ used in Eqns.~\ref{eqn:moosementum} \& \ref{eqn:temperature} and shown in the right panel of Fig.~\ref{fig:methods} is constructed as follows.
We read in the moose image in the left panel of Fig.~\ref{fig:methods}, and read the color value of each pixel.
We then compute a signed distance function $d(x,y)$ with $d(x,y) \in [-0.5, 0.5]$ to determine how far each pixel is from a boundary of the moose, with zero values being at the boundaries.
We next calculate the mask value of each pixel as
\begin{equation}
\mathcal{M}(x,y) = \frac{1}{2}\left(1-\mathrm{erf}\left[\frac{\sqrt{\pi} d(x,y)}{\delta}\right]\right),
\end{equation}
where we choose $\delta = 31.113 \left(\rm{Pr}/\rm{Ra}\right)^{1/4} \tilde{\gamma}^{-1/2}$ (see appendix \ref{app:nondim_equations}), which is ten times larger than the optimal, marginally-resolved $\delta$.
We then interpolate the mask and sample it on our simulation grid in polar ($r, \phi$) coordinates.
The resulting moose mask is used directly during timestepping.
For more specifics, we refer the reader to \url{https://github.com/evanhanders/moosinesq_convection/blob/main/masks/smooth_moosinesq_ibm.ipynb}.
