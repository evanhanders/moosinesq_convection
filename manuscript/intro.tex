\section{Introduction}
\label{sec:introduction}

Convection is important in all sorts of natural systems\footnote{Trust us, we're experts}.
In particular, most stars have convection zones.
This can be determined by looking at the Sun\footnote{Please do not look at the Sun.} or by running 1D stellar evolution models.
It is crucial that we gain a better understanding of convection since it is such a ubiquitous and poorly understood process in astrophysics\footnote{The financial wellfare of many of the authors depend on you agreeing with this statement.}.

Convective motions are three-dimensional, so many modern convective experiments utilize multi-dimensional fluid dynamical simulations.
However, modern computational resources are too limited to time-evolve the Navier-Stokes equations in their most complex form \citep{landau}, including all of the important physics present in stars \citep{Paxton2011, Paxton2013, Paxton2015, Paxton2018, Paxton2019}.
As a result, numericists make simplifying assumptions to create tractable experiments, for example by simplifying the geometry or the equations.
The most widely-used simplified set of equations is called the \emph{Boussinesq approximation} \citep{spiegel_veronis_1960}, and it has been used in thousands of studies \citep[see e.g.,][]{ahlers_etal_2009}.

In this Letter, we present the first ever simulations which use an oft-overlooked simplified equation set: the \emph{Moosinesq} approximation.
This approximation is similar to the Boussinesq approximation in every way, except it allows for the study of convection inside of a moose (\emph{Alces alces}).
While this approximation has many\footnote{\emph{many}} applications in the terrestrial context, the authors of this paper are largely astronomers.
Therefore we are most interested in the application of this approximation to \emph{moosive stars}, that is, stars with masses $M \gtrsim 1.3 M_\odot$ which have convective course on the main sequence.
In section \ref{sec:methods}, we describe the Moosinesq equations and our numerical methods.
In section \ref{sec:results}, we demonstrate the nature of flows in two-dimensional Moosinesq Convection.
Finally, in section \ref{sec:conclusions}, we discuss applications of Moosinesq Convection to stars, human society, and beyond.
