\section{Introduction}
\label{sec:introduction}

Convection is important in all sorts of natural systems\footnote{Trust us, we're experts}.
In particular, most stars have convection zones.
This can be determined by looking at the Sun\footnote{Please do not look at the Sun.} or by running 1D stellar evolution models.
It is crucial that we gain a better understanding of convection since it is such a ubiquitous and poorly understood process in astrophysics\footnote{The financial wellfare of many of the authors depends on you agreeing with this statement.}.

Convective motions are three-dimensional, so many modern experiments use multi-dimensional fluid dynamical simulations.
However, modern computational resources are too limited to time-evolve the Navier-Stokes equations in their most complex form \citep{landau}, a form required by all of the important physics present in stars \citep{Paxton2011, Paxton2013, Paxton2015, Paxton2018, Paxton2019}.
As a result, numericists make simplifying assumptions to create tractable experiments, for example by simplifying the geometry or the equations.
The most widely-used simplified set of equations employs the \emph{Boussinesq approximation}\footnote{Or the ``incompressible except for when it's not'' approximation, though this term has not caught on in the literature.} \citep{spiegel_veronis_1960}, which has been used in thousands of studies \citep[see e.g.,][]{ahlers_etal_2009}.
Under this approximation, buoyancy-driving density variations depend linearly on temperature, but flows are otherwise incompressible so there is no density stratification or sound waves.

In this Letter, we present the first ever simulations that use an oft-overlooked fluid approximation: the \emph{Moosinesq} approximation.
The Moosinesq and Boussinesq approximations are identical in every way, except convection inside of a moose (\emph{Alces alces}) can only be studied under the Moosinesq approximation.
This approximation is suitable for describing the active and dynamic inner lives and environments (both physical and mental; see \citealp{Gibson2015}) of the moose.
The moose is a large mammal indigenous to North America and Europe and can have a mass of up to $M_\mathrm{moose}\equiv 550$~kg and a vertical length scale of up to $L_\mathrm{moose} \equiv 2$~m \citep{CPWmoose}.
The word ``moose'' is seen in observations to be both singular and plural, a fact that has long eluded explanation by even the most talented theorists.
Our study is not the first time that moose have prompted significant scientific or technological development \citep[see, e.g.,][]{Handel2009}.
For a brief taste of the rich historical interactions between human and moose, we refer the reader to appendix~\ref{app:history}.

While Moosinesq convection has many\footnote{\emph{many}} applications in the terrestrial context, the authors of this Letter are mostly astrophysicists.
Therefore we focus on the application of this approximation to \emph{moosive stars}, that is, stars with masses $M \gtrsim 4\times 10^{27} \; M_\mathrm{moose}$ (or $1.1 \; M_\odot$) which have convective cores while on the main sequence.
In section \ref{sec:methods}, we describe the Moosinesq equations and our numerical methods.
In section \ref{sec:results}, we demonstrate the nature of flows in two-dimensional Moosinesq convection.
Finally, in section \ref{sec:conclusions}, we discuss applications of Moosinesq convection to stars, human society, and beyond.
