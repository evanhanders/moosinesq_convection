\section{Introduction}
\label{sec:introduction}

The Boussinesq approximation is a commonly-used approximation in the field of fluid dynamics [liberal citations of authors' work here].
It has [certain advantages] and [certain disadvantages], and is ultimately recognized for its great utility in [fluids jargon here].

However, the Boussinesq approximation must be abandoned in situations where compressibility is an important consideration.
This can occur in microbiology \citep{Ravetto2014}, and it has long been known that wildlife ecology is another such situation \citep[e.g.][]{Enright1963}.
Such studies have often focused on domains in which animal compressibility is more pronounced (i.e. deep-sea life).
However, motivated by a recent report of an unexpected occurrence of acute compression and deformation in a land-animal context \citep{Gudmannsson2018}---as well as a delightful linguistic coincidence---we present ground-breaking work on fundamental fluid dynamics in the context of the moose (\textit{Alces alces}), a domain which we dub the \textit{moosinesq approximation}.
This approximation is suitable for describing the active and dynamic inner lives and environments (both physical and mental; see \citealp{Gibson2015}) of the moose.

The moose is a large mammal indigenous to North America and Europe, which can have a mass of up to 550~kg and a height of up to 2~m \citep{CPWmoose}.
...
Our study is not the first time that moose have prompted significant scientific or technological development \citep[see, e.g.,][]{Handel2009}.
