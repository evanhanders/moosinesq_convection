\section{Historical context}
\label{app:history}

The circumpolar distribution of Alces alces (moose) has led to multiple independent points of cultural connection to and subsequent efforts directed at harnessing the largest cervid with varying levels of success. 
Rock carvings in the Kalbak-Tash group, Altai Republic, Russia indicate that ab antuquo efforts to ride or cause moose to pull sleds have been documented and subsequently received comment since the Bronze Age and likely earlier \citep{useev_2014}.
In North America, European efforts to colonize Canada have been intermittently aided by the capture of wild individual moose more suited to traversing long distances in snow and over boggy ground than domestic alternatives leading to practical applications such as the use of a team of moose to deliver mail by Mr. W.R. Day \citep{archives_unleashed}.
The exploitation of this practicality was limited, obviously, by the size and temperament of the moose which after about the age of 3 days is entirely antagonistic towards humans \citep{sipko_etal_2019}.
A few spectacular exceptions of moose under harness has kept hope burning for a future with a less fraught relationship. 
The most noteworthy partnership is widely remembered through the efforts of Albert Vaillancourt of Chelmsford, Ontario who exhibited moose pulling a surrey during the intermissions of horse racing \citep{chisholm_2019}.
His pair of racing mooses named Moose and Silver clearly demonstrated the majesty and potential of this species \citep{landry_1941}.
 
This potential inspires the continuing quest for a fully domestic moose that is somewhat less likely to attack and kill humans. 
In Russia, moose domestication is the subject of investigations initiated by Prof. P. A. Manteifel who oversaw an effort to create moose nurseries across Russia for the purpose of creating a recognizably domestic animal from about 1934 through the present \citep{sipko_etal_2019}.
In an early, perhaps premature demonstration on December 1937, I.V. Stalin watched a military moose drill. 
He was, “particularly impressed by the moment when moose cavalry flew out of the forest, bristling with machine guns.” He did note that the moose were not yet trained to distinguish the Red Army soldiers from the White Finns \citep{pererva_2017}.
